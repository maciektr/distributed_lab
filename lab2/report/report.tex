\documentclass{article}
\usepackage[utf8]{inputenc}
\usepackage{polski}
\usepackage{geometry}
\usepackage{pdfpages}
\usepackage{pdfpages}
\usepackage{listings}
\usepackage{listingsutf8}
\usepackage{multirow}
\usepackage{siunitx}
\usepackage{multirow}
\usepackage{booktabs}
\usepackage{tabularx}
\usepackage{placeins}
\usepackage{pdflscape}
\usepackage{graphicx}
\usepackage{subfig}
\usepackage{hyperref}
\usepackage{amsmath}
\usepackage{colortbl}

\geometry{
a4paper,
total={170mm,257mm},
left=20mm,
top=20mm
}
\newcolumntype{Y}{>{\centering\arraybackslash}X}
% \renewcommand\thesection{}
\lstset{%
literate=%
 {ą}{{\k{a}}}1
 {ę}{{\k{e}}}1
 {Ą}{{\k{A}}}1
 {Ę}{{\k{E}}}1
 {ś}{{\'{s}}}1
 {Ś}{{\'{S}}}1
 {ź}{{\'{z}}}1
 {Ź}{{\'{Z}}}1
 {ń}{{\'{n}}}1
 {Ń}{{\'{N}}}1
 {ć}{{\'{c}}}1
 {Ć}{{\'{C}}}1
 {ó}{{\'{o}}}1
 {Ó}{{\'{O}}}1
 {ż}{{\.{z}}}1
 {Ż}{{\.{Z}}}1
 {ł}{{\l{}}}1
 {Ł}{{\l{}}}1
}

\title{Systemy Rozproszone\\ 
Laboratorium II - RabbitMQ}
\author{Maciej Trątnowiecki}
\date{AGH, Semestr Letni, 2021}

\begin{document}
    \maketitle
    \lstset{ 
      backgroundcolor=\color{white},   % choose the background color; you must add \usepackage{color} or \usepackage{xcolor}; should come as last argument
      basicstyle=\footnotesize,        % the size of the fonts that are used for the code
      breakatwhitespace=false,         % sets if automatic breaks should only happen at whitespace
      breaklines=true,                 % sets automatic line breaking
      captionpos=b,                    % sets the caption-position to bottom
      commentstyle=\color{mygreen},    % comment style
      deletekeywords={...},            % if you want to delete keywords from the given language
      escapeinside={\%*}{*)},          % if you want to add LaTeX within your code
      %extendedchars=true,              % lets you use non-ASCII characters; for 8-bits encodings only, does not work with UTF-8
      firstnumber=1000,                % start line enumeration with line 1000
      frame=single,	                   % adds a frame around the code
      keepspaces=true,                 % keeps spaces in text, useful for keeping indentation of code (possibly needs columns=flexible)
      keywordstyle=\color{blue},       % keyword style
      language=Octave,                 % the language of the code
      morekeywords={*,...},            % if you want to add more keywords to the set
      numbers=left,                    % where to put the line-numbers; possible values are (none, left, right)
      numbersep=5pt,                   % how far the line-numbers are from the code
      numberstyle=\tiny\color{mygray}, % the style that is used for the line-numbers
      rulecolor=\color{black},         % if not set, the frame-color may be changed on line-breaks within not-black text (e.g. comments (green here))
      showspaces=false,                % show spaces everywhere adding particular underscores; it overrides 'showstringspaces'
      showstringspaces=false,          % underline spaces within strings only
      showtabs=false,                  % show tabs within strings adding particular underscores
      stepnumber=2,                    % the step between two line-numbers. If it's 1, each line will be numbered
      stringstyle=\color{mymauve},     % string literal style
      tabsize=2,	                   % sets default tabsize to 2 spaces
      title=\lstname                   % show the filename of files included with \lstinputlisting; also try caption instead of title
    }
    
    \section{Obserwacja mechanizmu niezawodności kolejki}
        W celu przeprowadzenia obserwacji przetestowałem działanie systemu dla trzech sposobów potwierdzeń otrzymania wiadomośći. 
        \begin{itemize}
            \item Potwierdzenia wysyłane przez RabbitMQ API po otrzymaniu wiadomości
            \item Potwierdzenia wysyłane ręcznie po przetworzeniu wiadomości 
            \item Brak potwierdzeń
        \end{itemize}
        Najwyższą niezawodność zapewniają potwierdzenia przesyłane po zakończeniu przetwarzania wiadomości. W ten sposób otrzymujemy system odporny na przerwanie przetwarzania otrzymanej już wiadomości w przypadku wystąpienia błędu - w tej sytuacji proces konsumenta będzie ponownie otrzymywał nieprzetworzoną wiadomość aż do otrzymania przez serwer kolejki potwierdzenia informującego o poprawnym zakończeniu przetwarzania. \\
        W przypadku automatycznego wysyłania potwierdzenia po odbiorze wiadomości serwer nie gwarantuje jej przetworzenia. To jest, w przypadku przerwania przetwarzania w wyniku błędu w procesie konsumenta, system nigdy nie spróbuje ponownie przetworzyć otrzymanej wiadomości. \\
        Jeśli nie zapewnimy przesyłania potwierdzeń w żaden sposób, serwer będzie ponownie wysyłał te same wiadomości do procesu konsumenta, nawet jeśli zostały już przetworzone kilkukrotnie. Każda nowa wiadomość będzie dodawana na koniec kolejki nieprzerwania przesyłanych wiadomości. 
        
    \section{}
        \lstinputlisting[language=bash]{no_qos.txt}
        \lstinputlisting[language=bash]{qos.txt}
        
\end{document}

    % \section{Pomiary czasu wykonania programu}
    %     \begin{itemize}
    %         \item Szerokość obrazu: 850
    %         \item Wysokość obrazu: 800
    %         \item Liczba iteracji: 1000
    %         \item Przybliżenie: 300
    %     \end{itemize}
    %     \begin{center}
    %         \begin{table}[ht]
    %             \centering
    %             \begin{tabular}{|c|c|c|}
    %                 \hline
    %                 Liczba wątków w egzekutorze  & Liczba podproblemów & Średni czas wykonania (milisekundy) \\
    %                 \specialrule{1pt}{1pt}{1pt}
    %                 1 & 1 & 790 \\
    %                 \hline
    %                 4 & 4 & 450 \\
    %                 \hline
    %                 8  & 4 & 450 \\
    %                 \hline 
    %                 8 & 8 & 360 \\
    %                 \hline
    %                 8 & 16 & 240\\
    %                 \hline
    %                 8 & 80 & 210\\
    %                 \hline
    %                 8 & 800 & 220\\
    %                 \hline 
    %                 16 & 16 & 240\\
    %                 \hline
    %                 16 & 32 & 220\\
    %                 \hline
    %                 16 & 160 & 210\\
    %                 \hline
    %                 \end{tabular}
    %             \caption{Pomiar czasów wykonania}
    %             \label{tab:my_label}
    %         \end{table}
    %     \end{center}